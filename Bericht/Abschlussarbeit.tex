\documentclass[12pt,a4paper,bibliography=totocnumbered,listof=totocnumbered]{scrartcl}
\usepackage[ngerman]{babel}
\usepackage[utf8]{inputenc}
\usepackage{amsmath}
\usepackage{amsfonts}
\usepackage{amssymb}
\usepackage{graphicx}
\usepackage{fancyhdr}
\usepackage{tabularx}
\usepackage{geometry}
\usepackage{setspace}
\usepackage[right]{eurosym}
\usepackage[printonlyused]{acronym}
\usepackage{subfig}
\usepackage{floatflt}
\usepackage[usenames,dvipsnames]{color}
\usepackage{colortbl}
\usepackage{paralist}
\usepackage{array}
\usepackage{titlesec}
\usepackage{parskip}
\usepackage[right]{eurosym}

\usepackage[subfigure,titles]{tocloft}
\usepackage[pdfpagelabels=true]{hyperref}

\usepackage{listings}
\lstset{basicstyle=\footnotesize, captionpos=b, breaklines=true, showstringspaces=false, tabsize=2, frame=lines, numbers=left, numberstyle=\tiny, xleftmargin=2em, framexleftmargin=2em}
\makeatletter
\def\l@lstlisting#1#2{\@dottedtocline{1}{0em}{1em}{\hspace{1,5em} Lst. #1}{#2}}
\makeatother

\geometry{a4paper, top=27mm, left=30mm, right=20mm, bottom=35mm, headsep=10mm, footskip=12mm}

\hypersetup{unicode=false, pdftoolbar=true, pdfmenubar=true, pdffitwindow=false, pdfstartview={FitH},
	pdftitle={Fallstudie GIT},
	pdfauthor={Vedad Hamamdzic},
	pdfsubject={Abschlussarbeit},
	pdfcreator={\LaTeX\ with package \flqq hyperref\frqq},
	pdfproducer={pdfTeX \the\pdftexversion.\pdftexrevision},
	pdfkeywords={Ausarbeitung},
	pdfnewwindow=true,
	colorlinks=true,linkcolor=black,citecolor=black,filecolor=magenta,urlcolor=black}
\pdfinfo{/CreationDate (D:20110620133321)}

\begin{document}

\titlespacing{\section}{0pt}{12pt plus 4pt minus 2pt}{-6pt plus 2pt minus 2pt}

% Kopf- und Fusszeile
\renewcommand{\sectionmark}[1]{\markright{#1}}
\renewcommand{\leftmark}{\rightmark}
\pagestyle{fancy}
\lhead{}
\chead{}
\rhead{\thesection\space\contentsname}
\lfoot{Fallstudie Entwicklungswerkzeuge: GIT\newline}
\cfoot{}
\rfoot{\ \linebreak Seite \thepage}
\renewcommand{\headrulewidth}{0.4pt}
\renewcommand{\footrulewidth}{0.4pt}

% Vorspann
\renewcommand{\thesection}{\Roman{section}}
\renewcommand{\theHsection}{\Roman{section}}
\pagenumbering{Roman}

% ----------------------------------------------------------------------------------------------------------
% Titelseite
% ----------------------------------------------------------------------------------------------------------
\thispagestyle{empty}
\begin{center}
	\includegraphics[scale=1]{Bilder/hs_os.png}\\
	\vspace*{2cm}
	\Large
	\textbf{Fallstudie Entwicklungswerkzeuge}\\
	\textbf{}\\
	\vspace*{2cm}
	\Huge
	\textbf{Ausarbeitung}\\
	\vspace*{0.5cm}
	\large
	über das Thema\\
	\vspace*{1cm}
	\textbf{GIT Versionsverwaltungssystem}\\
	\vspace*{2cm}
	
	\vfill
	\normalsize
	\newcolumntype{x}[1]{>{\raggedleft\arraybackslash\hspace{0pt}}p{#1}}
	\begin{tabular}{x{6cm}p{7.5cm}}
		\rule{0mm}{5ex}\textbf{Autor:} & Vedad Hamamdzic\newline email@email.de \\ 
		\rule{0mm}{5ex}\textbf{Prüfer:} & Paul Layer \\ 
		\rule{0mm}{5ex}\textbf{Abgabedatum:} & 18.11.2014 \\ 
	\end{tabular} 
\end{center}
\pagebreak

% ----------------------------------------------------------------------------------------------------------
% Abstract
% ----------------------------------------------------------------------------------------------------------
\setcounter{page}{1}
\onehalfspacing
\titlespacing{\section}{0pt}{12pt plus 4pt minus 2pt}{2pt plus 2pt minus 2pt}
\rhead{KURZFASSUNG}
\section{Zusammenfassung}

Lorem ipsum dolor sit amet, consetetur sadipscing elitr, sed diam nonumy eirmod tempor invidunt ut labore et dolore magna aliquyam erat, sed diam voluptua. At vero eos et accusam et justo duo dolores et ea rebum. Stet clita kasd gubergren, no sea takimata sanctus est Lorem ipsum dolor sit amet. Lorem ipsum dolor sit amet, consetetur sadipscing elitr, sed diam nonumy eirmod tempor invidunt ut labore et dolore magna aliquyam erat, sed diam voluptua. At vero eos et accusam et justo duo dolores et ea rebum. Stet clita kasd gubergren, no sea takimata sanctus est Lorem ipsum dolor sit amet.

\vspace{-1,2em}
\titlespacing{\section}{0pt}{12pt plus 4pt minus 2pt}{-6pt plus 2pt minus 2pt}
\section*{Abstract}
Das ganze auf Englisch.
\pagebreak

% ----------------------------------------------------------------------------------------------------------
% Verzeichnisse
% ----------------------------------------------------------------------------------------------------------
% TODO Typ vor Nummer
\renewcommand{\cfttabpresnum}{Tab. }
\renewcommand{\cftfigpresnum}{Abb. }
\settowidth{\cfttabnumwidth}{Abb. 10\quad}
\settowidth{\cftfignumwidth}{Abb. 10\quad}

\titlespacing{\section}{0pt}{12pt plus 4pt minus 2pt}{2pt plus 2pt minus 2pt}
\singlespacing
\rhead{INHALTSVERZEICHNIS}
\renewcommand{\contentsname}{II Inhaltsverzeichnis}
\phantomsection
\addcontentsline{toc}{section}{\texorpdfstring{II \hspace{0.35em}Inhaltsverzeichnis}{Inhaltsverzeichnis}}
\addtocounter{section}{1}
\tableofcontents
\pagebreak
\rhead{VERZEICHNISSE}
\listoffigures
\pagebreak
\listoftables
%\pagebreak
\renewcommand{\lstlistlistingname}{Listing-Verzeichnis}
{\labelsep2cm\lstlistoflistings}
\pagebreak

% ----------------------------------------------------------------------------------------------------------
% Abkürzungen
% ----------------------------------------------------------------------------------------------------------
\section{Abkürzungsverzeichnis}
\begin{acronym}[OSGi] % längste Abkürzung steht in eckigen Klammern
	\setlength{\itemsep}{-\parsep} % geringerer Zeilenabstand
	\acro{OSGi}{Open Service Gateway initiative}
\end{acronym}
\newpage

% ----------------------------------------------------------------------------------------------------------
% Inhalt
% ----------------------------------------------------------------------------------------------------------
% Abstände Überschrift
\titlespacing{\section}{0pt}{12pt plus 4pt minus 2pt}{-6pt plus 2pt minus 2pt}
\titlespacing{\subsection}{0pt}{12pt plus 4pt minus 2pt}{-6pt plus 2pt minus 2pt}
\titlespacing{\subsubsection}{0pt}{12pt plus 4pt minus 2pt}{-6pt plus 2pt minus 2pt}

% Kopfzeile
\renewcommand{\sectionmark}[1]{\markright{#1}}
\renewcommand{\subsectionmark}[1]{}
\renewcommand{\subsubsectionmark}[1]{}
\lhead{Kapitel \thesection}
\rhead{\rightmark}

\onehalfspacing
\renewcommand{\thesection}{\arabic{section}}
\renewcommand{\theHsection}{\arabic{section}}
\setcounter{section}{0}
\pagenumbering{arabic}
\setcounter{page}{1}

% ----------------------------------------------------------------------------------------------------------
% Einleitung
% ----------------------------------------------------------------------------------------------------------
\section{GIT}

\subsection{Was ist GIT}
GIT ist ein Versionsverwaltungssystem. Doch was ist ein Versionsverwaltungssystem? Ein Versionsverwaltungssystem ist ein System, welches Änderungen an einer Datei oder eine Reihe von Dateien protokolliert, so dass  bestimmte Versionen später wieder aufrufbar sind.\footnote{\cite{chacon2009pro} Seite 1 Zeile 1 } Um Problemen entgegenzuwirken die eine Amateurhafte Methoden der Versionsverwaltung mit sich bringen wie z.B das ständige kopieren neuer Versionen in ein Verzeichnis. Wurden diese Systeme entwickelt.Dabei unterscheidet man 3 Arten von Systemen. Der wesentlichste Unterschied besteht darin wie und wo die Daten gehalten werden.
\subsubsection{Lokale Versionskontrollsysteme}

Von Lokalen Versionskontrollsystemen spricht man wenn die Daten auf dem Lokalen System vorligen (siehe Abbildung 1 ). Dabei werden die Dateien in einer Version Database (Repository) gehalten. Nach jedem Checkout wird automatisch eine neue Version im Repository erstellt. Somit entgeht man der Gefahr durch das oben erwähnte Kopieren in andere Verzeichnisse eine der Versionen zu überschreiben, da man vergessen hat die Datei umzubenennen. Natürlich ist diese Variante der Versionskontrolle für Große Projekte die im Team bearbeitet werden eher destruktiv.Ein Beispiel für Lokale Systeme ist RCS( Revision Control System ).Eher geeignet für Teamwork sind die beiden andern Architekturen. 
\newline
\vspace{3pt}
\begin{minipage}{\linewidth}
	\centering
	\includegraphics[width=0.4\linewidth]{Bilder/LVKS.png}
	\captionof{figure}[Lokale Architektur]{Lokale Architektur  \cite{chacon2009pro}\footnotemark }
	\label{fig:osgi}
\end{minipage}
\newpage
\subsubsection{Zentralisierte Versionskontrollsysteme}
Bei zentralisierte Versionskontrollsystemen wird die Versionierung nicht lokal vorgenommen. Die Entwickler haben einen Zentralen Punkt(Abbildung 2) , einen Server und dort befindet sich der Quellcode des Projektes in einem Reposytory zu deutsch Lager. Der unterschied zu einfachen Lokalen Systemen ist nun Offensichtlich. Man braucht zumindest ein Netzwerk um solche Systeme zu nutzen. Ein sehr beliebtes zentralisiertes System ist Subversion. Ein weiterer Vorteil gegenüber der lokalen Versionnierung besteht darin das gemeinsames Arbeiten an einem Projekt möglich ist und bei Verwendung eines Servers der Online erreichbar ist kann das Arbeiten auch ohne Ortsbindung ablaufen. Doch dieser Vorteil der Ortsungebundenheit bietet einen enormen „Single Point of Failure“ denn wenn der Server ausfällt ist man nicht in der Lage seiner Arbeit nachzugehen.

\newline
\vspace{3pt}
\begin{minipage}{\linewidth}
	\centering
	\includegraphics[width=0.3\linewidth]{Bilder/sub.png}
	\captionof{figure}[Zentralisierte Architektur]{Zentralisierte Architektur\footnotemark }
	\label{fig:osgi}
\end{minipage}    

\subsubsection{Verteilte Versionskontrollsysteme}
GIT gehört zu den Verteilten Systemen, der Unterschied zu den Varianten davor ist das sie beides können. 
Einer Art hybride Lösung. Man ist in der Lage Lokal zu Versionieren aber auch im Netzwerk Versionen, anderen zur Verfügung zu stellen (Abbildung 3). Jeder kann als Server fungieren und somit  wird der „Single Point of Failure“ eliminiert den Zentralisierten Systeme haben. In der Praxis ist aber eher üblich das man einen Server nutzt vor allem bei Teamarbeiten. Wenn dieser jedoch ausfällt ist man in der Lage weiter seine Arbeit zu verrichten.
\newline
\vspace{3pt}
\begin{minipage}{\linewidth}
	\centering
	\includegraphics[width=0.3\linewidth]{Bilder/git.png}
	\captionof{figure}[Verteilte Architektur]{Verteilte Architektur\footnotemark }
	\label{fig:osgi}
\end{minipage}
   

%\vspace{1em}
%\begin{minipage}{\linewidth}
%	\centering
%	\includegraphics[width=0.7\linewidth]{Bilder/layering-osgi.png}
%	\captionof{figure}[OSGi Architektur]{OSGi Architektur\footnotemark }
%	\label{fig:osgi}
%\end{minipage}
%\footnotetext{Quelle: \url{http://www.osgi.org/Technology/WhatIsOSGi}}

\subsection{GIT Historie }
Im Jahre 2005 ist es zu unstimmigkeiten zwichen der Etwicklercommunety von Linux und dem Anbieter des proprietären BitKeeper-Systems das vorher kostenfrei genutz wurde. Die Linux-Kernel-Entwickler mussten sich was einfallen lassen.Deswegen begann  Linus Torvalds im April 2005 mit der Entwicklung von GIT und präsentierte auch sehr schnell die erste Version. Git baute auf den Erfahrungen mit BitKeeper auf doch die Hauptziele des neuen Systems waren\footnote{\cite{chacon2009pro} Seite 5}:

\begin{compactitem}
	\item Geschwindigkeit
	\item Einfaches Design
	\item Gute Unterstützung von nicht-linearer Entwicklung (tausende paralleler verschiedener Verzweigungen der Versionen)
	\item Vollständig verteilt
	\item Fähig, große Projekte wie den Linux Kernel effektiv zu verwalten
\end{compactitem}
Durch die kontinuierliche Weiterentwicklung des Systems und die Benutzerfreundlichkeit wurde Git zu einen sehr beliebten Tool.
Ein großen Einfluss auf den Erfolg von Git hat auch die Social Coding Plattform GitHub auf der man viele Open Source Projekte findet wie z.B :

\begin{compactitem}
	\item Der Linux Kernel\footnote{\cite{linux}}
	\item Ruby on Rails\footnote{\cite{ruby}}
	\item Die Javascript Bibliothek JQuery \footnote{\cite{jquery}}
	\item Das CMS Joomla\footnote{\cite{joomla}}
\end{compactitem}

Das sind natürlich nicht alle Open Source Projekte die GIT in Verbindung mit GitHub nutzen, aber einige bekannte die sich für Git entschieden haben. Der Dienst, den GitHub bereitstellt ist kostenfrei, doch nur unter der Bedingung das die Projekte öffentlich zugänglich sind. Des Weiteren gibt es Optional wählbare Services die gegen Bezahlung verfügbar sind, aber auch eine Enterprise version die für Firmen interessant sein kann wird bereitgestellt.    
\pagebreak



\section{Anwendungsgebiete von GIT }
Für Auflistungen wird die \textit{compactitem}-Umgebung genutzt, wodurch der Zeilenabstand zwischen den Punkten verringert wird.

\begin{compactitem}
	\item Nur
	\item ein
	\item Beispiel.
\end{compactitem}

\subsection{Listings}
Zuletzt ein Beispiel für ein Listing, in dem Quellcode eingebunden werden kann, siehe Listing \ref{lst:arduino}.

\vspace{1em}
\begin{lstlisting}[caption=Arduino Beispielprogramm, label=lst:arduino]
int ledPin = 13;
void setup() {
    pinMode(ledPin, OUTPUT);
}
void loop() {
    digitalWrite(ledPin, HIGH);
    delay(500);
    digitalWrite(ledPin, LOW);
    delay(500);
}
\end{lstlisting}

In diesem Abschnitt wird eine Tabelle (siehe Tabelle \ref{tab:beispiel}) dargestellt.

\vspace{1em}
\begin{table}[!h]
	\centering
	\begin{tabular}{|l|l|l|}
		\hline
		\textbf{Name} & \textbf{Name} & \textbf{Name}\\
		\hline
		1 & 2 & 3\\
		\hline
		4 & 5 & 6\\
		\hline
		7 & 8 & 9\\
		\hline
	\end{tabular}
	\caption{Beispieltabelle}
	\label{tab:beispiel}
\end{table}

\subsection{Tipps}
Die Quellen befinden sich in der Datei \textit{bibo.bib}. Ein Buch- und eine Online-Quelle sind beispielhaft eingefügt. [Vgl. \cite{buch}, \cite{online}]

Abkürzungen lassen sich natürlich auch nutzen (\ac{OSGi}). Weiter oben im Latex-Code findet sich das Verzeichnis.
\pagebreak

% ----------------------------------------------------------------------------------------------------------
% Kapitel
% ----------------------------------------------------------------------------------------------------------
\section{GIT Grundlagen}
Lorem ipsum dolor sit amet.
\subsection{Begriffe die man kennen sollte GIT}
Bevor man mit Git Arbeitet sollte man einige begriffe kennenlernen, damit man auch weiss was man Tut.

\subsection{Installation von GIT}
Lorem ipsum dolor sit amet, consetetur sadipscing elitr, sed diam nonumy eirmod tempor invidunt ut labore et dolore magna aliquyam erat, sed diam voluptua. At vero eos et accusam et justo duo dolores et ea rebum. Stet clita kasd gubergren, no sea takimata sanctus est Lorem ipsum dolor sit amet. Lorem ipsum dolor sit amet, consetetur sadipscing elitr, sed diam nonumy eirmod tempor invidunt ut labore et dolore magna aliquyam erat, sed diam voluptua. At vero eos et accusam et justo duo dolores et ea rebum. Stet clita kasd gubergren, no sea takimata sanctus est Lorem ipsum dolor sit amet.

\subsection{Konfiguration von GIT}
Lorem ipsum dolor sit amet, consetetur sadipscing elitr, sed diam nonumy eirmod tempor invidunt ut labore et dolore magna aliquyam erat, sed diam voluptua. At vero eos et accusam et justo duo dolores et ea rebum. Stet clita kasd gubergren, no sea takimata sanctus est Lorem ipsum dolor sit amet. Lorem ipsum dolor sit amet, consetetur sadipscing elitr, sed diam nonumy eirmod tempor invidunt ut labore et dolore magna aliquyam erat, sed diam voluptua. At vero eos et accusam et justo duo dolores et ea rebum. Stet clita kasd gubergren, no sea takimata sanctus est Lorem ipsum dolor sit amet.

\subsection{Hilfestellungen durch das System}
Lorem ipsum dolor sit amet, consetetur sadipscing elitr, sed diam nonumy eirmod tempor invidunt ut labore et dolore magna aliquyam erat, sed diam voluptua. At vero eos et accusam et justo duo dolores et ea rebum. Stet clita kasd gubergren, no sea takimata sanctus est Lorem ipsum dolor sit amet. Lorem ipsum dolor sit amet, consetetur sadipscing elitr, sed diam nonumy eirmod tempor invidunt ut labore et dolore magna aliquyam erat, sed diam voluptua. At vero eos et accusam et justo duo dolores et ea rebum. Stet clita kasd gubergren, no sea takimata sanctus est Lorem ipsum dolor sit amet.
\pagebreak

% ----------------------------------------------------------------------------------------------------------
% Kapitel 
% ----------------------------------------------------------------------------------------------------------
\section{KAp}
Lorem ipsum dolor sit amet.

\subsection{Unterkapitel}
Lorem ipsum dolor sit amet, consetetur sadipscing elitr, sed diam nonumy eirmod tempor invidunt ut labore et dolore magna aliquyam erat, sed diam voluptua. At vero eos et accusam et justo duo dolores et ea rebum. Stet clita kasd gubergren, no sea takimata sanctus est Lorem ipsum dolor sit amet. Lorem ipsum dolor sit amet, consetetur sadipscing elitr, sed diam nonumy eirmod tempor invidunt ut labore et dolore magna aliquyam erat, sed diam voluptua. At vero eos et accusam et justo duo dolores et ea rebum. Stet clita kasd gubergren, no sea takimata sanctus est Lorem ipsum dolor sit amet.

\subsection{Unterkapitel}
Lorem ipsum dolor sit amet, consetetur sadipscing elitr, sed diam nonumy eirmod tempor invidunt ut labore et dolore magna aliquyam erat, sed diam voluptua. At vero eos et accusam et justo duo dolores et ea rebum. Stet clita kasd gubergren, no sea takimata sanctus est Lorem ipsum dolor sit amet. Lorem ipsum dolor sit amet, consetetur sadipscing elitr, sed diam nonumy eirmod tempor invidunt ut labore et dolore magna aliquyam erat, sed diam voluptua. At vero eos et accusam et justo duo dolores et ea rebum. Stet clita kasd gubergren, no sea takimata sanctus est Lorem ipsum dolor sit amet.
\pagebreak

% ----------------------------------------------------------------------------------------------------------
% Kapitel
% ----------------------------------------------------------------------------------------------------------
\section{Kapitel}
Lorem ipsum dolor sit amet.

\subsection{Unterkapitel}
Lorem ipsum dolor sit amet, consetetur sadipscing elitr, sed diam nonumy eirmod tempor invidunt ut labore et dolore magna aliquyam erat, sed diam voluptua. At vero eos et accusam et justo duo dolores et ea rebum. Stet clita kasd gubergren, no sea takimata sanctus est Lorem ipsum dolor sit amet. Lorem ipsum dolor sit amet, consetetur sadipscing elitr, sed diam nonumy eirmod tempor invidunt ut labore et dolore magna aliquyam erat, sed diam voluptua. At vero eos et accusam et justo duo dolores et ea rebum. Stet clita kasd gubergren, no sea takimata sanctus est Lorem ipsum dolor sit amet.

\subsection{Unterkapitel}
Lorem ipsum dolor sit amet, consetetur sadipscing elitr, sed diam nonumy eirmod tempor invidunt ut labore et dolore magna aliquyam erat, sed diam voluptua. At vero eos et accusam et justo duo dolores et ea rebum. Stet clita kasd gubergren, no sea takimata sanctus est Lorem ipsum dolor sit amet. Lorem ipsum dolor sit amet, consetetur sadipscing elitr, sed diam nonumy eirmod tempor invidunt ut labore et dolore magna aliquyam erat, sed diam voluptua. At vero eos et accusam et justo duo dolores et ea rebum. Stet clita kasd gubergren, no sea takimata sanctus est Lorem ipsum dolor sit amet.
\pagebreak

% ----------------------------------------------------------------------------------------------------------
% Literatur
% ----------------------------------------------------------------------------------------------------------
\renewcommand\refname{Quellenverzeichnis}
\bibliographystyle{myalpha}
\bibliography{bibo}
\pagebreak

% ----------------------------------------------------------------------------------------------------------
% Anhang
% ----------------------------------------------------------------------------------------------------------
\pagenumbering{Roman}
\setcounter{page}{1}
\lhead{Anhang \thesection}

\begin{appendix}
\section*{Anhang}
\phantomsection
\addcontentsline{toc}{section}{Anhang}
\addtocontents{toc}{\vspace{-0.5em}}

\section{GUI}
Ein toller Anhang.

\subsection*{Screenshot}
\label{app:screenshot}
Unterkategorie, die nicht im Inhaltsverzeichnis auftaucht.

\end{appendix}


\newpage
\thispagestyle{empty}
\begin{center}
	\vspace*{5em}
	\huge\textbf{Erklärung}\\
\end{center}
\vspace{2em}
Hiermit versichere ich, dass ich meine Abschlussarbeit selbständig verfasst und keine anderen als die angegebenen Quellen und Hilfsmittel benutzt habe.

\vspace{4em}
\begin{minipage}{\linewidth}
	\begin{tabular}{p{15em}p{15em}}
		Datum: &  .......................................................\\
		& \centering (Unterschrift)\\
	\end{tabular}
\end{minipage}

\end{document}
